% !TEX TS-program = pdflatex
% !TEX encoding = UTF-8 Unicode

% This is a simple template for a LaTeX document using the "article" class.
% See "book", "report", "letter" for other types of document.

\documentclass[11pt]{article} % use larger type; default would be 10pt

\usepackage[english]{babel}
\usepackage[utf8]{inputenc} % set input encoding (not needed with XeLaTeX)
\usepackage{fvextra}
\usepackage{csquotes}

%%% Examples of Article customizations
% These packages are optional, depending whether you want the features they provide.
% See the LaTeX Companion or other references for full information.

%%% PAGE DIMENSIONS
\usepackage[a4paper, top=2cm, bottom=2.2cm, left=3cm, right=3cm, includeheadfoot, headheight=31pt,
    headsep=0pt]{geometry} % to change the page dimensions
\geometry{a4paper} % or letterpaper (US) or a5paper or....
\usepackage{multicol}
\setlength{\headheight}{35pt}
\setlength{\parindent}{0pt}
% \geometry{landscape} % set up the page for landscape
%   read geometry.pdf for detailed page layout information

\usepackage{graphicx} % support the \includegraphics command and options
\usepackage[colorlinks=true, allcolors=blue]{hyperref}
\usepackage{xcolor} % Allow coloring of text.

% \usepackage[parfill]{parskip} % Activate to begin paragraphs with an empty line rather than an indent

%%% PACKAGES
\usepackage{booktabs} % for much better looking tables
\usepackage{array} % for better arrays (eg matrices) in maths
\usepackage{paralist} % very flexible & customisable lists (eg. enumerate/itemize, etc.)
\usepackage{fancyvrb} % adds environment for commenting out blocks of text & for better verbatim
\usepackage{subfig} % make it possible to include more than one captioned figure/table in a single float
\usepackage{float}
\usepackage{biblatex}[style=ieeetr]
\usepackage{parskip}
\usepackage{amsmath, amsthm, amssymb, amsfonts}
\usepackage{lipsum}
\usepackage{tabularx}
\usepackage{calculator}
\bibliography{refs}
% These packages are all incorporated in the memoir class to one degree or another...

% Changes date format to YYYY-MM-DD
\usepackage[yyyymmdd]{datetime}
\settimeformat{hhmmsstime}
\renewcommand{\dateseparator}{-}

% Add version number variable.
\newcounter{versionNumber}
\makeatletter
% At the end write the current value back to the `.aux` file
\AtEndDocument{
    \immediate\write\@auxout{
        \string\setcounter{versionNumber}{\number\value{versionNumber}}
    }
}
\makeatother
% Step the counter at the beginning
\AtBeginDocument{
    \stepcounter{versionNumber}
}

% Title page info
\title{Rotating Index Load Balancer Pre-Production Specification}
\author{Lani Wagner\\\href{mailto:lani.wagner@students.fhnw.ch}{lani.wagner@students.fhnw.ch}}
\date{\today\ \currenttime} % Activate to display a given date or no
% otherwise the current date is printed

% Define constants to use throughout document
\makeatletter
\let\thisauthor\@author
\makeatother

%%% HEADERS & FOOTERS
\usepackage{lastpage}
\usepackage{fancyhdr} % This should be set AFTER setting up the page geometry
\pagestyle{fancy} % options: empty , plain , fancy
\renewcommand{\headrulewidth}{0pt} % customise the layout...
\lhead{}\chead{}\rhead{\includegraphics[height=0.88\headheight]{img/fhnw_ht_10mm.eps}}
\lfoot{v0.0.\theversionNumber\ \textcolor{gray}{\today\ \currenttime}}\cfoot{\thepage\ /
        {\hypersetup{hidelinks}\pageref{LastPage}}}\rfoot{Lani Wagner}

%%% SECTION TITLE APPEARANCE
\usepackage{sectsty}
\allsectionsfont{\sffamily\mdseries\upshape} % (See the fntguide.pdf for font help)
% (This matches ConTeXt defaults)

%%% ToC (table of contents) APPEARANCE
\usepackage[nottoc]{tocbibind} % Put the bibliography in the ToC
\usepackage[titles,subfigure]{tocloft}
\usepackage{amssymb} % Alter the style of the Table of Contents
\renewcommand{\cftsecfont}{\rmfamily\mdseries\upshape}
\renewcommand{\cftsecpagefont}{\rmfamily\mdseries\upshape} % No bold!

% Change second level of enumerated lists to arabic numbers.
\renewcommand{\labelenumii}{\arabic{enumi}.\arabic{enumii}}

% Reduce spacing in whole document to compact document
\usepackage{titlesec}
\usepackage{amsfonts}
\usepackage{enumitem}% http://ctan.org/pkg/enumitem
\setlist[itemize]{noitemsep, topsep=0pt}


%%% END Article customizations

%%% The "real" document content comes below...

\begin{document}
    \VerbatimFootnotes
    \clearpage\maketitle
    \thispagestyle{empty}
    \begin{abstract}
        This document contains the pre-production specification written for an implementation of a rotating index
        load balancer to reduce the I/O load for a service that is tasked with handling a large amount of requests,
        so that the I/O bottleneck is greatly reduced. This specification was written for the FHNW module Engineering
        Writing (engw).
    \end{abstract}
    {\hypersetup{hidelinks} \tableofcontents}
    \newpage



    \section{General}

    The result of this specification will be a library. This document documents the features this library must provide. This specification is purposefully \textit{not} programming-language-specific to allow the implementation in any language that best suits the existing infrastructure. The terminology in this document will be leaning on the terms most often used in while working with the java programming-language, such as type definitions. When working in another language the appropriate equivalent must be used in consideration of applicability and performance.

    Any specifications derived from referenced documents are overridden by any specifications in this document.



    \section{Design}

    The library shall provide an instantiable Singleton\textsuperscript{\ref{tab:glossary}} class named
    \verb|RotationBalancer<T>| with the following arguments. The required arguments must be implemented in the final
    product, but the passing of the arguments for the creation of the object is not required. The type \verb|T| or
    \verb|<T>| is a generic type that is set for every analysis when that request is made.

    \begin{table}[H]
        \centering
        \begin{tabular}{p{.1\linewidth} | p{.2\linewidth} | p{.2\linewidth} | p{.4\linewidth}}
            \textbf{Required} & \textbf{Type} & \textbf{Name} & \textbf{Content}
            \\\hline
            Yes & \verb|List<Path>| & \verb|folders| & The folders that should be scanned.
            \\\hline
            No & \verb|List<Path>| & \verb|excludedFolders| &
            Subfolders of
            \verb|folders|\hyperref[fn:1]{\footnotemark[1]} that don't need to be scanned. \\\hline
            No & \verb|Predicate<Path>| & \verb|isFileRelevant| & Executable function to verify whether the file is
            relevant\hyperref[fn:2]{\footnotemark[2]} for the analysis.
        \end{tabular}
        \caption{Library class arguments.}
        \label{tab:lib_args}
    \end{table}
    \footnotetext[1]{This must be verified when the class instance is created.}\label{fn:1}
    \footnotetext[2]{This f.e. allows for the exclusion of irrelevant files that may be stored in the same
    location. If no function is passed every file must be treated as relevant.}\label{fn:2}
    \addtocounter{footnote}{2}

    The instance shall only scan for files as long as active requests are being processed. This means that the scan
    should not start as soon as the class is instantiated. If the last request has been completed the balancer
    shall go into a sleeping state and woken up when a new request arrives.

    Every request to the API must be sent to an existing instance. That means instantiating the object before
    starting any analysis is critical. Once the object exists the user must be able to call the API with the following
    arguments and receive a response of type \verb|List<T>|.

    \begin{table}[H]
        \centering
        \begin{tabular}{p{.1\linewidth} | p{.2\linewidth} | p{.2\linewidth} | p{.4\linewidth}}
            \textbf{Required} & \textbf{Type} & \textbf{Name} & \textbf{Content}
            \\\hline
            Yes & \verb|Function<Path, T>| & \verb|analyzer| & The analysis function applied to every
            relevant\hyperref[fn:2]{\footnotemark[2]} file. \\\hline
            No & \verb|List<Path>| & \verb|excludedFolders| &
            Subfolders of
            \verb|folders|\hyperref[fn:1]{\footnotemark[1]} that don't need to be scanned. \\\hline
            No & \verb|Predicate<Path>| & \verb|isFileRelevant| & Executable function to verify whether the file is
            relevant\hyperref[fn:2]{\footnotemark[2]} for the analysis.
        \end{tabular}
        \caption{Library analysis call arguments.}
        \label{tab:call_args}
    \end{table}



    \section{Functional and Non-Functional Features}

    The contractor shall implement the following functional and non-functional features.
    \begin{itemize}
        \item[\ref{sec:3.1}] \hyperref[sec:3.1]{File Tree}
        \item[\ref{sec:3.2}] \hyperref[sec:3.2]{Handling Unpredictability}
        \item[\ref{sec:3.3}] \hyperref[sec:3.3]{Logging}
        \item[\ref{sec:3.4}] \hyperref[sec:3.4]{Code Style}
        \item[\ref{sec:3.5}] \hyperref[sec:3.5]{Testing}
        \item[\ref{sec:3.6}] \hyperref[sec:3.6]{Documentation}
    \end{itemize}


    \subsection{File Tree}\label{sec:3.1}

    To talk about \hyperref[sec:3.1]{unpredictability}, the term of file tree and how the file tree must be defined, is crucial. The file tree has a root, which is the folder that is passed to the library on initialization.\\If multiple folders were passed on initialization the root is an imaginary folder that is above all folders specified. Any files and folders at the same depth level in the tree are to be sorted in ascending lexicographical order according to their \href{https://en.wikipedia.org/wiki/Code_point}{Unicode codepoint} value.


    \subsection{Handling Unpredictability}\label{sec:3.2}

    This library must be able to work in an environment where files behave totally unpredictablly, due to external system or user interaction. The following possibilities and their respective appropriate responses are defined here.

    To handle unpredictability the term ``rotation'' needs to be defined. A rotation from any point has a starting file. A complete rotation goes from the starting file (inclusive) to the last file still in the file tree and then from the first file in the next file tree scan to the last file before the original starting file.


    \paragraph{A file disappears/is deleted mid-read}\label{par:file_disappear} ~\\
    The user should not notice anything about this. Meaning that any exceptions get caught and the file is not analyzed.


    \subparagraph{A file disappears/is deleted mid-analysis} ~\\
    This case does not include the possibility that a file disappears mid-analysis. If a file disappears during the analysisthe analysis should keep going undisturbed. We are interested about the results of the analysis at the time of read, not what it might be at the end of the analysis.


    \paragraph{All files (and folders) are deleted}\label{par:everything_deleted} ~\\
    If the read operation is interrupted by a file disappearing, that file is ignored as stated in the case of \hyperref[par:file_disappear]{``A file disappears/is deleted mid-read''} and the next file relevant for analysis is searched. If there are no further files to analyze the current result is returned.


    \subparagraph{All files (and folders) are replaced by a completely new set of files and folders} ~\\
    As stated in the paragraph \hyperref[par:file_disappear]{``A file disappears/is deleted mid-read''} the current file is ignored for analysis if a read operation is interrupted.


    \paragraph{The file where the analysis of a certain request started isn't around anymore on completion of one rotation\textsuperscript{\ref{tab:glossary}}} ~\\
    If the starting file cannot be found, reaching any parent folder\footnote{This includes parent folders at any depth. Closer parent folders take precendence.} means reaching the end of a rotation in this case. Parent folders include parent folders at any depth in the tree in upward direction (considering the root of the file system to be at the top). Folders further away from the root folder take precedence.

    If no parent folder can be found, all analysis data from the previous traverse of the file tree is discarded. The analysis then starts anew from the first file of the file tree and ends with the last file in the file tree.\\
    If the starting file was the first file in the file tree and it can't be found once the rest of the file tree was traversed the rotation is considered completed.


    \subsection{Logging}\label{sec:3.3}

    The following log-levels are the only ones allowed in this library. They are a subset of the log levels specified in the enum \href{https://docs.oracle.com/javase/8/docs/api/java/util/logging/class-use/Level.html}{\texttt{java.util.logging.Level}} and must be used in the following order (most granular first):

    \begin{enumerate}
        \item Fine
        \item Info
        \item Warning
        \item Severe
    \end{enumerate}

    Every single log message must include the date and time (including milliseconds) in the \href{https://en.wikipedia.org/wiki/ISO_8601}{ISO 8601} format. The following actions/events must be logged at the specified log level.

    \begin{table}[H]
        \centering
        \begin{tabular}{p{.2\linewidth} | p{.5\linewidth} | p{.2\linewidth}}
            \textbf{Event}                                             & \textbf{Metadata}                                            & \textbf{Loglevel} \\\hline
            Request received                                           & unique request identifier, request source, starting file     & Info              \\\hline
            File is visited (does not necessarily mean that it's read) & count of analyses for which this file is relevant                     & Fine              \\\hline
            File disappears mid-read                                   & file path, count of analyses for which this file is relevant                     & Info              \\\hline
            File tree traverse started                                 & first file path, count of active analyses in queue           & Info              \\\hline
            File tree traverse ended                                   & last file path, count of active analyses in queue            & Info              \\\hline
            Analysis data of a request is discarded                    & unique request identifier                                    & Info              \\\hline
            No files in file tree                                      &                                                              & Warning
        \end{tabular}
        \caption{Actions that must be logged including the metadata specified in their respective log-level.}
        \label{tab:log_actions}
    \end{table}


    \subsection{Code Style}\label{sec:3.4}

    All code must adhere to the \href{https://google.github.io/styleguide/javaguide.html}{Google Java Style Guide}.


    \subsection{Testing}\label{sec:3.5}

    The library must be tested with, at the bare minimum, an automated test for every paragraph in section 3.2 and every log event in 3.3. The logs to test section 3.3 need not arise from a functionality of the code, but may be tested by faking log statements.


    \subsection{Documentation}\label{sec:3.6}

    No additional documentation -- outside of the code, that is -- must be provided. The code must be documented using JavaDoc or the relevant equivalent in the implementation language. Every part of the public interface must be documented.

    \begin{itemize}
        \item Classes
        \begin{itemize}
            \item Interfaces
            \item Enums
        \end{itemize}
        \item Functions
    \end{itemize}

    Each documented artifact must contain the following information for

    \begin{itemize}
        \item[] classes\footnote{Do not document \href{https://www.oracle.com/technical-resources/articles/java/javadoc-tool.html#anonymous}{anonymous inner classes}.}:
        \begin{itemize}
            \item Short description
        \end{itemize}
        \item[] functions:
        \begin{itemize}
            \item Short description
            \item Parameters using the \verb|@param| tag
            \item Return type using the \verb|@return| tag \footnote{if not \verb|void|}
            \item Exceptions using the \verb|@exception| tag \footnote{as specified in \href{https://www.oracle.com/technical-resources/articles/java/javadoc-tool.html#throwstag}{Documenting Exceptions with @throws Tag}}
        \end{itemize}
    \end{itemize}

    Tags must be ordered as specified in the section \href{https://www.oracle.com/technical-resources/articles/java/javadoc-tool.html#orderoftags}{Order of Tags} in \href{https://www.oracle.com/technical-resources/articles/java/javadoc-tool.html}{How To Write Doc Comments for the Javadoc Tool}

    \newpage



    \section{Glossary}

    \begin{table}[H]
        \centering
        \begin{tabular}{p{.3\linewidth} | p{.6\linewidth}}
            \textbf{Term} & \textbf{Definition}
            \\\hline
            Singleton & A class that can only exist once in memory. Trying to create a new instance of the class
            should return the existing instance. \\\hline
            Rotation      & A single go around the list of files considered for analysis \\\hline
            Starting file & The file an analysis starts with.
        \end{tabular}
        \caption{Glossary definitions.}
        \label{tab:glossary}
    \end{table}
    \printbibliography[heading=bibintoc]
    \listoffigures
    \listoftables
\end{document}